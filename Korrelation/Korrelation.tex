\documentclass[a4paper,12pt]{article}
\usepackage{fancyhdr}
\usepackage{fancyheadings}
\usepackage[ngerman,german]{babel}
\usepackage{german}
\usepackage[utf8]{inputenc}
%\usepackage[latin1]{inputenc}
\usepackage[active]{srcltx}
\usepackage{algorithm}
\usepackage[noend]{algorithmic}
\usepackage{amsmath}
\usepackage{amssymb}
\usepackage{amsthm}
\usepackage{bbm}
\usepackage{enumerate}
\usepackage{graphicx}
\usepackage{ifthen}
\usepackage{listings}
\usepackage{struktex}
\usepackage{hyperref}
\usepackage{tikz}
\usetikzlibrary{positioning,automata}

%%%%%%%%%%%%%%%%%%%%%%%%%%%%%%%%%%%%%%%%%%%%%%%%%%%%%%
%%%%%%%%%%%%%% EDIT THIS PART %%%%%%%%%%%%%%%%%%%%%%%%
%%%%%%%%%%%%%%%%%%%%%%%%%%%%%%%%%%%%%%%%%%%%%%%%%%%%%%
\newcommand{\Fach}{Statistik}
\newcommand{\Name}{Robin Feldmann}
\newcommand{\Tutorium}{}
\newcommand{\Semester}{OHM}
\newcommand{\KlausurLoesung }{SoSe2020} %  <-- UPDATE ME
%%%%%%%%%%%%%%%%%%%%%%%%%%%%%%%%%%%%%%%%%%%%%%%%%%%%%%
%%%%%%%%%%%%%%%%%%%%%%%%%%%%%%%%%%%%%%%%%%%%%%%%%%%%%%

\setlength{\parindent}{0em}
\topmargin -1.0cm
\oddsidemargin 0cm
\evensidemargin 0cm
\setlength{\textheight}{9.2in}
\setlength{\textwidth}{6.0in}

%%%%%%%%%%%%%%%
%% Aufgaben-COMMAND
\newcommand{\Aufgabe}[1]{
  {
  \vspace*{0.5cm}
  \textsf{\textbf{Aufgabe #1}}
  \vspace*{0.2cm}
  
  }
}

\newcommand{\Definition}[1]{
  {
  \vspace*{0.5cm}
  \textsf{\textbf{#1}}
  \vspace*{0.2cm}
  
  }
}

%%%%%%%%%%%%%%
\hypersetup{
    pdftitle={\Fach{}: Übungsblatt \KlausurLoesung{}},
    pdfauthor={\Name},
    pdfborder={0 0 0}
}

\lstset{ %
language=java,
basicstyle=\footnotesize\tt,
showtabs=false,
tabsize=2,
captionpos=b,
breaklines=true,
extendedchars=true,
showstringspaces=false,
flexiblecolumns=true,
}

\title{Übungsblatt Deterministische Endliche Automaten}
\author{\Name{}}

\begin{document}
\thispagestyle{fancy}
\lhead{\sf \large \Fach{} \\ \small \Name{} }
\rhead{\sf \Semester{} \\  Tutorium\Tutorium{}}
\vspace*{0.2cm}
\begin{center}
\LARGE \sf \textbf{Korrelation}
\end{center}
\vspace*{0.2cm}

%%%%%%%%%%%%%%%%%%%%%%%%%%%%%%%%%%%%%%%%%%%%%%%%%%%%%%
%% Insert your solutions here %%%%%%%%%%%%%%%%%%%%%%%%
%%%%%%%%%%%%%%%%%%%%%%%%%%%%%%%%%%%%%%%%%%%%%%%%%%%%%%
\Definition{Kovarianz}
Für zwei dimensionale Daten $(x_1,y_1),...,(x_n,y_n)$ ist die empirische Kovarianz definiert als: 
$$
cov(x,y) = s_{xy} = \frac{1}{n} \sum^n_{i=1}(x_i - \bar{x})(y_i - \bar{y}) = \bar{xy} - \bar{x}\cdot \bar{y} 
$$

\Definition{Korrelationskoeffizient}
Für zwei dimensionale Daten $(x_1,y_1),...,(x_n,y_n)$ ist der Korrelationskoeffizient definiert als: 
$$
r_{xy} = \frac{\sum^n_{i=1}(x_i - \bar{x})(y_i - \bar{y})}{\sqrt{\sum^n_{i=1}(x_i - \bar{x})^2 \cdot \sum^n_{i=1}(y_i - \bar{y})^2}} = \frac{s_{xy}}{\sigma_x \sigma_y}
$$

\Aufgabe{1}
Hängt die Entfernung vom Stadtzentrum mit den Mietpreisen zusammen?
$$
\begin{tabular}[h]{l|lllll}
Miete in Euro/$ m^2$ & 23 & 17 & 13 & 9  & 7\\
\hline
Entfernung in km & 0& 5 &10 & 15 & 20 \\
\end{tabular}
$$
Korrelieren die Daten? Geben sie eine Schätzung ab!
Berechnen sie Mittelwerte, Kovarianz und Korrelation.

\Aufgabe{2}
Vorsicht bei der Interpreration:
Hängt die Anzahl der existierenden Piraten mit der globalen Durchschnittstemperatur zusammen?$$
\begin{tabular}[h]{l|llll}
Jahr & 1820 & 1880 & 1940 & 2000 \\
\hline
Anzahl Piraten (in tausend)& 35 & 20 & 5 & 0.02 \\
\hline
Globale Durschnittstemp in C & 14.2 & 14.7 &15.2 & 16 \\
\end{tabular}
$$
Korrelieren die Daten? Geben sie eine Schätzung ab!
Berechnen sie Mittelwerte, Kovarianz und Korrelation.


\Aufgabe{3}
Statistische Umfragen unter Frisören ergaben folgende Werte.
$$
\begin{tabular}[h]{l|llll}
Alter in Jahren & 20 & 36 & 56 & 73 \\
\hline
Graue Haare in \% & 0& 15 & 46 & 83 \\
\end{tabular}
$$
Korrelieren die Daten? Geben sie eine Schätzung ab!
Berechnen sie Mittelwerte, Kovarianz und Korrelation.

\Aufgabe{4}
Statistische Umfragen unter Frisören ergaben folgende Werte.
$$
\begin{tabular}[h]{l|llll}
Alter in Jahren & 20 & 36 & 56 & 73 \\
\hline
Anzahl Haare (in Tausend) &  100 & 75 & 50 & 20 \\
\end{tabular}
$$
Korrelieren die Daten? Geben sie eine Schätzung ab!
Berechnen sie Mittelwerte, Kovarianz und Korrelation.

\Aufgabe{5}
Statistische Umfragen unter Frisören ergaben folgende Werte.
$$
\begin{tabular}[h]{l|llll}
Alter in Jahren & 20 & 36 & 56 & 73 \\
\hline
Flausen im Kopf &  42 & 13 & 11 & 65 \\
\end{tabular}
$$
Korrelieren die Daten? Geben sie eine Schätzung ab!
Berechnen sie Mittelwerte, Kovarianz und Korrelation.




\Aufgabe{6}
Gegeben sei die folgende zweidimensionale Stichprobe zur Abhängigkeit der Dichte D (in Gramm pro Liter) von der Temperatur T (in Grad Celsius) bei trockener Luft:
$$
\begin{tabular}[h]{l|lllll}
T & -20 & -10 & 0 & 10 & 20 \\
\hline
D & 1.39 & 1.34 & 1.29 & 1.25 & 1.20 \\
\end{tabular}
$$
Berechne den Korrelationskoeffizienten $r_{TD}$.
 (SoSe16 Aufgabe 2).
 


%%%%%%%%%%%%%%%%%%%%%%%%%%%%%%%%%%%%%%%%%%%%%%%%%%%%%%
%%%%%%%%%%%%%%%%%%%%%%%%%%%%%%%%%%%%%%%%%%%%%%%%%%%%%%
\end{document}

