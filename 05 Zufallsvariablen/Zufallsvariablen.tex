\documentclass[a4paper,12pt]{article}
\usepackage{fancyhdr}
\usepackage{fancyheadings}
\usepackage[ngerman,german]{babel}
\usepackage{german}
\usepackage[utf8]{inputenc}
%\usepackage[latin1]{inputenc}
\usepackage[active]{srcltx}
\usepackage{algorithm}
\usepackage[noend]{algorithmic}
\usepackage{amsmath}
\usepackage{amssymb}
\usepackage{amsthm}
\usepackage{bbm}
\usepackage{enumerate}
\usepackage{graphicx}
\usepackage{ifthen}
\usepackage{listings}
\usepackage{struktex}
\usepackage{hyperref}
\usepackage{tikz}
\usetikzlibrary{positioning,automata}

%%%%%%%%%%%%%%%%%%%%%%%%%%%%%%%%%%%%%%%%%%%%%%%%%%%%%%
%%%%%%%%%%%%%% EDIT THIS PART %%%%%%%%%%%%%%%%%%%%%%%%
%%%%%%%%%%%%%%%%%%%%%%%%%%%%%%%%%%%%%%%%%%%%%%%%%%%%%%
\newcommand{\Fach}{Statistik}
\newcommand{\Name}{Robin Feldmann}
\newcommand{\Tutorium}{}
\newcommand{\Semester}{OHM}
\newcommand{\KlausurLoesung }{SoSe2020} %  <-- UPDATE ME
%%%%%%%%%%%%%%%%%%%%%%%%%%%%%%%%%%%%%%%%%%%%%%%%%%%%%%
%%%%%%%%%%%%%%%%%%%%%%%%%%%%%%%%%%%%%%%%%%%%%%%%%%%%%%

\setlength{\parindent}{0em}
\topmargin -1.0cm
\oddsidemargin 0cm
\evensidemargin 0cm
\setlength{\textheight}{9.2in}
\setlength{\textwidth}{6.0in}

%%%%%%%%%%%%%%%
%% Aufgaben-COMMAND
\newcommand{\Aufgabe}[1]{
  {
  \vspace*{0.5cm}
  \textsf{\textbf{Aufgabe #1}}
  \vspace*{0.2cm}
  
  }
}

\newcommand{\Definition}[1]{
  {
  \vspace*{0.5cm}
  \textsf{\textbf{#1}}
  \vspace*{0.2cm}
  
  }
}

%%%%%%%%%%%%%%
\hypersetup{
    pdftitle={\Fach{}: Übungsblatt \KlausurLoesung{}},
    pdfauthor={\Name},
    pdfborder={0 0 0}
}

\lstset{ %
language=java,
basicstyle=\footnotesize\tt,
showtabs=false,
tabsize=2,
captionpos=b,
breaklines=true,
extendedchars=true,
showstringspaces=false,
flexiblecolumns=true,
}

\title{Übungsblatt Deterministische Endliche Automaten}
\author{\Name{}}

\begin{document}
\thispagestyle{fancy}
\lhead{\sf \large \Fach{} \\ \small \Name{} }
\rhead{\sf \Semester{} \\  Tutorium\Tutorium{}}
\vspace*{0.2cm}
\begin{center}
\LARGE \sf \textbf{Zufallsvariablen}
\end{center}
\vspace*{0.2cm}

%%%%%%%%%%%%%%%%%%%%%%%%%%%%%%%%%%%%%%%%%%%%%%%%%%%%%%
%% Insert your solutions here %%%%%%%%%%%%%%%%%%%%%%%%
%%%%%%%%%%%%%%%%%%%%%%%%%%%%%%%%%%%%%%%%%%%%%%%%%%%%%%
\Definition{Zufallsvariable}
Eine Abbildung X, mit:
$$ X: \Omega \rightarrow \mathbb{R}, \quad \omega \rightarrow X(\omega)$$
heißt Zufallsvariable auf $\Omega$.

\Definition{Verteilungsfunktion}
Sei X eine Zufallsvariable auf $\Omega$ dann heißt die Abbildung:
$$ F: \mathbb{R} \rightarrow [0;1 ], \quad F:=\mathbb{P}(X(\omega)\leq x) =: \mathbb{P}(X\leq x)$$
Verteilungsfunktion F von X. Eine Verteilungsfunktion ist immer monoton wachsend:
$$ x_1 \le x_2 \Rightarrow F(x_1) \le F(x_2) $$
und es gilt:
$$ \lim\limits_{x \to -\infty} F(x) = 0, \quad \lim\limits_{x \to \infty} F(x) = 1 $$

\Definition{Diskrete Zufallsvariablen: Wahrscheinlichkeitsfunktion}
Die Wahrscheinlichkeitsfunktion gibt die Wahrscheinlichkeit einzelner Ausprägungen an:
$$f(x) := \mathbb{P}(X=x) =  \lim\limits_{h \to^>0} F(x + h ) -   \lim\limits_{h \to^>0} F(x - h ) $$

\Definition{Stetige Zufallsvariablen: Wahrscheinlichkeitsdichte}
Wahrscheinlichkeitsdichte ist die Ableitung der Verteilungsfunktion:
$$f(x) := F(x)'$$ 
und hat die Eigenschaften:
$$ f(x) \geq 0, \quad  \int_{-\infty}^{\infty} f(x)dx = 1$$

\Definition{Interpretation}
Die Wahrscheinlichkeit, dass die Zufallsvariable X einen Wert kleiner oder gleich $a$ annimmt:
$$ \mathbb{P}(X \leq  a)= F(a) =  \int_{-\infty}^{a} f(x)dx$$
Die Wahrscheinlichkeit, dass die Zufallsvariable X einen Wert größer $a$ annimmt:
$$ \mathbb{P}(a < X)=  1 -   \mathbb{P}(X \leq  a)  = 1- F(a) = 1 -  \int_{-\infty}^{a} f(x)dx$$
Die Wahrscheinlichkeit, dass die Zufallsvariable X einen Wert zwischen $a$ und $b$ annimmt:
$$ \mathbb{P}(a < X \leq b)= F(b) - F(a) = \int_{a}^{b} f(x)dx$$


\Aufgabe{1}
Bei einem Münzwurfspiel kann man Geld gewinnen. Es werden zwei Münzen geworfen und die Werte addiert. Kopf entspricht dabei 0 und Zahl 1.
Bei einer Summe von 0 verliert der Spieler 1€, bei einer Summe von 1 gibt es weder Verlust noch Gewinn. Bei einer Summe von 2 gewinnt der Spieler 1€.
Bestimme und zeichne Verteilungsfunktion und Wahrscheinlichkeitsfunktion.

\Aufgabe{2}
Es sei X eine stetige Zufallsvariable, mit Verteilungsfunktion:
\begin{equation}
  F(x) = \begin{cases}
     0 & \text{f"ur } x \leq 0 \\
      \frac{x}{10}  & \text{f"ur } 0  < x \leq 10 \\
    1 & \text{sonst }  
   \end{cases}
\end{equation}
Zeichne die Verteilungsfunktion. Bestimme und zeichne die zugehörige Dichtefunktion $f$. Berechne $\mathbb{P}(X \leq 10), \mathbb{P}(X \leq 5),\mathbb{P}(X > 10),\mathbb{P}(5 < X \leq 7)$.

\Aufgabe{3}
Die Dichte f einer stetigen Zufallsgröße X sei gegeben durch 
\begin{equation}
   f(x) = \begin{cases}
     \alpha & \text{f"ur } 1 \leq x \le 3 \\
     \frac{1}{2} & \text{f"ur } 3 \leq x \le 4\\
    0 & \text{sonst }  
   \end{cases}
\end{equation}
Bestimme $ \alpha$ und skizziere $f$ .
Bestimme und skizziere die Verteilungsfunktion $F$ von $X$.
Berechne $\mathbb{P}(X \leq 3)$. \\

\Aufgabe{4}
Bei einem Würfelspiel gewinnt bzw. verliert der Spieler immer die Differenz zur Vier. Würfelt der Spieler zum Beispiel eine Vier gewinnt er 0€, würfelt er eine 1 so verliert er 3€, würfelt er aber eine 6 so gewinnt er 2€. Bestimme und zeichne Verteilungsfunktion und Wahrscheinlichkeitsfunktion.



\Aufgabe{5}
Es sei X eine stetige Zufallsvariable, mit Verteilungsfunktion:
\begin{equation}
  F(x) = \begin{cases}
     0 & \text{f"ur } x \leq 1 \\
      \frac{x}{4} -\frac{1}{4}    & \text{f"ur } 1  < x \leq 3 \\
       \frac{x}{2}  - 1  & \text{f"ur } 3 < x \leq 4 \\
    1 & \text{sonst }  
   \end{cases}
\end{equation}
Zeichne die Verteilungsfunktion. Bestimme und zeichne die zugehörige Dichtefunktion $f$. Berechne $\mathbb{P}(X \leq 2), \mathbb{P}(X \leq 3),\mathbb{P}(X > 3),\mathbb{P}(3 < X \leq 4)$.

\Aufgabe{6}
Es sei X eine stetige Zufallsvariable, für die gilt:
\begin{equation}
   \mathbb{P}(X >x) = \begin{cases}
     x^{-3} & \text{f"ur } x > 1 \\
    1 & \text{sonst }  
   \end{cases}
\end{equation}
Berechnen Sie die zugehörige Dichtefunktion f.
(SoSe20)


\Aufgabe{7}
Die Dichte f einer stetigen Zufallsgröße X sei gegeben durch 
\begin{equation}
   f(x) = \begin{cases}
     \alpha & \text{f"ur } 0 \leq x \le 2 \\
    0 & \text{sonst }  
   \end{cases}
\end{equation}
Bestimme $ \alpha$ und skizziere $f$ .
Bestimme und skizziere die Verteilungsfunktion $F$ von $X$.
Berechne $\mathbb{P}(X \leq 1)$. \\





\Aufgabe{8}
Die Dichte f einer stetigen Zufallsgröße X sei gegeben durch 
\begin{equation}
   f(x) = \begin{cases}
     \alpha & \text{f"ur } 0 \leq x \le 4 \\
     \frac{1}{4} & \text{f"ur } 5 \leq x \le 7 \\
    0 & \text{sonst }  
   \end{cases}
\end{equation}
Bestimme $ \alpha$ und skizziere $f$ . (Ergebnis: $\alpha = \frac{1}{8}$) \\
Bestimme und skizziere die Verteilungsfunktion $F$ von $X$. \\
 (SoSe16).


%%%%%%%%%%%%%%%%%%%%%%%%%%%%%%%%%%%%%%%%%%%%%%%%%%%%%%
%%%%%%%%%%%%%%%%%%%%%%%%%%%%%%%%%%%%%%%%%%%%%%%%%%%%%%
\end{document}

