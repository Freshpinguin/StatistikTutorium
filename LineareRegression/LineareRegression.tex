\documentclass[a4paper,12pt]{article}
\usepackage{fancyhdr}
\usepackage{fancyheadings}
\usepackage[ngerman,german]{babel}
\usepackage{german}
\usepackage[utf8]{inputenc}
%\usepackage[latin1]{inputenc}
\usepackage[active]{srcltx}
\usepackage{algorithm}
\usepackage[noend]{algorithmic}
\usepackage{amsmath}
\usepackage{amssymb}
\usepackage{amsthm}
\usepackage{bbm}
\usepackage{enumerate}
\usepackage{graphicx}
\usepackage{ifthen}
\usepackage{listings}
\usepackage{struktex}
\usepackage{hyperref}
\usepackage{tikz}
\usetikzlibrary{positioning,automata}

%%%%%%%%%%%%%%%%%%%%%%%%%%%%%%%%%%%%%%%%%%%%%%%%%%%%%%
%%%%%%%%%%%%%% EDIT THIS PART %%%%%%%%%%%%%%%%%%%%%%%%
%%%%%%%%%%%%%%%%%%%%%%%%%%%%%%%%%%%%%%%%%%%%%%%%%%%%%%
\newcommand{\Fach}{Statistik}
\newcommand{\Name}{Robin Feldmann}
\newcommand{\Tutorium}{}
\newcommand{\Semester}{OHM}
\newcommand{\KlausurLoesung }{SoSe2020} %  <-- UPDATE ME
%%%%%%%%%%%%%%%%%%%%%%%%%%%%%%%%%%%%%%%%%%%%%%%%%%%%%%
%%%%%%%%%%%%%%%%%%%%%%%%%%%%%%%%%%%%%%%%%%%%%%%%%%%%%%

\setlength{\parindent}{0em}
\topmargin -1.0cm
\oddsidemargin 0cm
\evensidemargin 0cm
\setlength{\textheight}{9.2in}
\setlength{\textwidth}{6.0in}

%%%%%%%%%%%%%%%
%% Aufgaben-COMMAND
\newcommand{\Aufgabe}[1]{
  {
  \vspace*{0.5cm}
  \textsf{\textbf{Aufgabe #1}}
  \vspace*{0.2cm}
  
  }
}

\newcommand{\Definition}[1]{
  {
  \vspace*{0.5cm}
  \textsf{\textbf{#1}}
  \vspace*{0.2cm}
  
  }
}

%%%%%%%%%%%%%%
\hypersetup{
    pdftitle={\Fach{}: Übungsblatt \KlausurLoesung{}},
    pdfauthor={\Name},
    pdfborder={0 0 0}
}

\lstset{ %
language=java,
basicstyle=\footnotesize\tt,
showtabs=false,
tabsize=2,
captionpos=b,
breaklines=true,
extendedchars=true,
showstringspaces=false,
flexiblecolumns=true,
}

\title{Übungsblatt Deterministische Endliche Automaten}
\author{\Name{}}

\begin{document}
\thispagestyle{fancy}
\lhead{\sf \large \Fach{} \\ \small \Name{} }
\rhead{\sf \Semester{} \\  Tutorium\Tutorium{}}
\vspace*{0.2cm}
\begin{center}
\LARGE \sf \textbf{Regression}
\end{center}
\vspace*{0.2cm}

%%%%%%%%%%%%%%%%%%%%%%%%%%%%%%%%%%%%%%%%%%%%%%%%%%%%%%
%% Insert your solutions here %%%%%%%%%%%%%%%%%%%%%%%%
%%%%%%%%%%%%%%%%%%%%%%%%%%%%%%%%%%%%%%%%%%%%%%%%%%%%%%
\Definition{Lösungsskizze}
Legen sie mit Hilfe der Methode der kleinsten quadratischen Abweichung von Gauß eine Kurve der Form
$$
y = f_{a,b}(x) 
$$
durch die Punkte $(x_1,y_1),....,(x_n,y_n)$. Bestimmen sie also a und b.
\\

Bestimme die quadratische Abweichung:
$$
Q(a,b) = \sum_{i=1}^n (y_i - f_{a,b}(x_i))^2
$$
Nach a und b ableiten. Dann gleich Null setzen und so a und b berechnen.
$$
\frac{dQ}{da} Q(a,b)=0 
$$
$$
\frac{dQ}{db} Q(a,b)=0
$$
\\
\\
Falls $ f_{a,b}(x)= a + b\cdot x $:
\\



$$b=\frac{cov(x,y)}{var(x)} \text{ und } a= \bar{y} - b \cdot \bar{x} 
$$
\\
Zur Erinnerung: 
$$
cov(x,y) = s_{xy} = \frac{1}{n} \sum^n_{i=1}(x_i - \bar{x})(y_i - \bar{y}) = \bar{xy} - \bar{x}\cdot \bar{y} 
$$
$$
Var(  \vec{x}) = \frac{1}{n} \sum^n_{i=1} (x_i- \bar{x} )^2 = \bar{x^2} - \bar{x}^2 
$$

\Aufgabe{1}
$$
\begin{tabular}[h]{l|lll}
x  & 2 & 1& 3 \\
\hline
y& 2 & 1 & 1.5 \\
\end{tabular}
$$
Berechnen Sie eine Gerade$f(x)= a + b\cdot x$ die bezüglich der kleinsten Quadrate diese Punkte optimal annähert.
Berechnen Sie für diese Funktion sowie die folgenden die quadratische  Abweichung.
\begin{enumerate}[a)]
\item
$y= \frac{1}{3}2^x$
\item
$y=-x^2 +4x - 2 $
\end{enumerate}
Für eine Kurve $y= a\cdot 2^x$ soll der Parameter a bezüglich der quadratischen Abweichung als optimal bestimmt werden.


\Aufgabe{2}
Gegeben sei die folgende zweidimensionale Stichprobe zur Abhängigkeit der Dichte D (in Gramm pro Liter) von der Temperatur T (in Grad Celsius) bei trockener Luft:
$$
\begin{tabular}[h]{l|lllll}
T & -20 & -10 & 0 & 10 & 20 \\
\hline
D & 1.39 & 1.34 & 1.29 & 1.25 & 1.20 \\
\end{tabular}
$$
Man bestimme die Ausgleichsgerade $D=aT + b$, d.h. die Koeffizienten a und b.
 (SoSe16 Aufgabe 2).
 
\Aufgabe{3}
\begin{enumerate}[a)]
\item Legen Sie mit Hilfe der Methode der kleinsten quadratischen Abweichung von Gauß eine Kurve der Form $y=a\cdot 2^x$ durch die drei Punkte $(0,2),(1,4),(2,13)\in \mathbb{R}^2$. Bestimmen Sie also a.
\item Legt man mit derselben Methode eine Kurve der Form $y=c\cdot x$ durch obige drei Punkte, so erhält man $c=6$. Welche der beiden Kurven passt besser im Sinne der quadratischen Abweichung?
\end{enumerate}
(Aus WiSe 14/15) 


\Aufgabe{4}
\begin{enumerate}[a)]
\item Legen Sie mit Hilfe der Methode der kleinsten quadratischen Abweichung von Gauß eine Kurve der Form $y=a\cdot 3^x$ durch die drei Punkte $(0,1),(1,4),(2,13)\in \mathbb{R}^2$. Bestimmen Sie also a.
\item Legt man mit derselben Methode eine Kurve der Form $y=c\cdot x$ durch obige drei Punkte, so erhält man $c=6$. Welche der beiden Kurven passt besser im Sinne der quadratischen Abweichung?
\end{enumerate}
(Aus WiSe 15/16) 

\Aufgabe{5}
Bei einer Messung ergaben sich für die x-Werte $x_1 =1, x_2=2, x_3=3 $ und für die y-Werte $y_1=3, y_2=7, y_3=10$.
\begin{enumerate}[a)]
\item
Legen Sie mit Hilfe der Gaußeschen Methode der kleinsten quadratischen Abweichung eine Kurve der Form $y=ax^2 + 2$ durch diese Punkte. Berechnen Sie also a.
\item
Legt man mit der selben Methode eine Kurve der Form $y=bx+2$ durch diese Punkte, so ergibt sich $b=2.5$. Welche der beiden Kurven passt im Sinne der kleinsten quadratischen Abweichung besser? Begründung! 
\end{enumerate}
(Aus SoSe18 Aufgabe 1.)

\Aufgabe{6}
Bei einer Messung ergaben sich für die x-Werte
$$
x_1= -2, x_2=-1, x_3=1, x_4=2
$$
und für die y-Werte
$$
y_1=5, y_2=4, y_3=1 \text{ und } y_4=7
$$
Legen sie mit Hilfe der Gaußschen Methode der kleinsten quadratischen Abweichung eine Kurve der Form $y=ax^3 +b$ durch diese Punkte. Bestimmen sie also die Parameter a und b.


%%%%%%%%%%%%%%%%%%%%%%%%%%%%%%%%%%%%%%%%%%%%%%%%%%%%%%
%%%%%%%%%%%%%%%%%%%%%%%%%%%%%%%%%%%%%%%%%%%%%%%%%%%%%%
\end{document}

