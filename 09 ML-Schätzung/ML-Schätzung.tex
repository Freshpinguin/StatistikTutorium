\documentclass[a4paper,12pt]{article}
\usepackage{fancyhdr}
\usepackage{fancyheadings}
\usepackage[ngerman,german]{babel}
\usepackage{german}
\usepackage[utf8]{inputenc}
%\usepackage[latin1]{inputenc}
\usepackage[active]{srcltx}
\usepackage{algorithm}
\usepackage[noend]{algorithmic}
\usepackage{amsmath}
\usepackage{amssymb}
\usepackage{amsthm}
\usepackage{bbm}
\usepackage{enumerate}
\usepackage{graphicx}
\usepackage{ifthen}
\usepackage{listings}
\usepackage{struktex}
\usepackage{hyperref}
\usepackage{tikz}
\usetikzlibrary{positioning,automata}

%%%%%%%%%%%%%%%%%%%%%%%%%%%%%%%%%%%%%%%%%%%%%%%%%%%%%%
%%%%%%%%%%%%%% EDIT THIS PART %%%%%%%%%%%%%%%%%%%%%%%%
%%%%%%%%%%%%%%%%%%%%%%%%%%%%%%%%%%%%%%%%%%%%%%%%%%%%%%
\newcommand{\Fach}{Statistik}
\newcommand{\Name}{Robin Feldmann}
\newcommand{\Tutorium}{}
\newcommand{\Semester}{OHM}
\newcommand{\KlausurLoesung }{SoSe2020} %  <-- UPDATE ME
%%%%%%%%%%%%%%%%%%%%%%%%%%%%%%%%%%%%%%%%%%%%%%%%%%%%%%
%%%%%%%%%%%%%%%%%%%%%%%%%%%%%%%%%%%%%%%%%%%%%%%%%%%%%%

\setlength{\parindent}{0em}
\topmargin -1.0cm
\oddsidemargin 0cm
\evensidemargin 0cm
\setlength{\textheight}{9.2in}
\setlength{\textwidth}{6.0in}

%%%%%%%%%%%%%%%
%% Aufgaben-COMMAND
\newcommand{\Aufgabe}[1]{
  {
  \vspace*{0.5cm}
  \textsf{\textbf{Aufgabe #1}}
  \vspace*{0.2cm}
  
  }
}

\newcommand{\Definition}[1]{
  {
  \vspace*{0.5cm}
  \textsf{\textbf{#1}}
  \vspace*{0.2cm}
  
  }
}

%%%%%%%%%%%%%%
\hypersetup{
    pdftitle={\Fach{}: Übungsblatt \KlausurLoesung{}},
    pdfauthor={\Name},
    pdfborder={0 0 0}
}

\lstset{ %
language=java,
basicstyle=\footnotesize\tt,
showtabs=false,
tabsize=2,
captionpos=b,
breaklines=true,
extendedchars=true,
showstringspaces=false,
flexiblecolumns=true,
}

\title{Übungsblatt Deterministische Endliche Automaten}
\author{\Name{}}

\begin{document}
\thispagestyle{fancy}
\lhead{\sf \large \Fach{} \\ \small \Name{} }
\rhead{\sf \Semester{} \\  Tutorium\Tutorium{}}
\vspace*{0.2cm}
\begin{center}
\LARGE \sf \textbf{ML-Schätzung}
\end{center}
\vspace*{0.2cm}

%%%%%%%%%%%%%%%%%%%%%%%%%%%%%%%%%%%%%%%%%%%%%%%%%%%%%%
%% Insert your solutions here %%%%%%%%%%%%%%%%%%%%%%%%
%%%%%%%%%%%%%%%%%%%%%%%%%%%%%%%%%%%%%%%%%%%%%%%%%%%%%%
Gegeben ist eine Zufallsvariable X mit einer Verteilung $f_{\lambda}(x)$ die von dem Parameter $\lambda$ abhängt und eine Stichprobe unabhängiger Messwerte $x_1, .... x_n$. Ziel: Bestimme den Parameter $\lambda$ so, dass die Stichprobe am Wahrscheinlichsten ist. 



\Definition{Wahrscheinlichkeit der Stichprobe}
$$L(\lambda) = f_{\lambda}(x_1)\cdot f_{\lambda}(x_2)\cdot .... \cdot f_{\lambda}(x_n) $$

\Definition{Maximum der Wahrscheinlichkeit}
Ableiten und Null setzen.
$$ L(\lambda)' = 0 $$

\Definition{Log-Likelihood-Trick}

$$ l(\lambda) = ln( L(\lambda)) $$ 

Ableiten und Null setzen geht oft einfacher und gibt das gleiche Ergebnis, da der Logarithmus streng monoton steigend ist.
$$ l(\lambda)' = 0 $$

 
 \Aufgabe{1}
Die stetige Zufallsvariable X besitze die Dichte:
\begin{equation}
  f(x) = \begin{cases}
    \lambda \cdot e^{-\lambda x} & \text{f"ur } x > 0 \\
    0 & \text{sonst }  
   \end{cases}
\end{equation}
Eine Stichprobe ergab für X die Werte $x_1=13,x_2=9, x_3=8,x_4=10$.
Berechnen Sie den Maximum-Likeilhood Schätzer für $\lambda$.


\Aufgabe{2}
Die stetige Zufallsvariable X besitze die Dichte:
\begin{equation}
  f(x) = \begin{cases}
    0 & \text{f"ur } x \leq 0 \\
    c\cdot (x+1)^{-c-1} & \text{sonst }  
   \end{cases}
\end{equation}
Mit $c > 0$.
Eine Stichprobe ergab für X die Werte $x_1 = 3, x_2=1, x_3= 7$. Berechnen Sie den Maximum-Likeilhood Schätzer für c. (WiSe15/16)


\Aufgabe{3}
Eine stetige Zufallsgröße X habe die Dichte
\begin{equation}
  f(x) = \begin{cases}
   \lambda^5 \cdot \frac{x^4}{4!} \cdot e^{-\lambda x}& \text{f"ur } x \geq 0 \\
    0 & \text{sonst }  
   \end{cases}
\end{equation}
Man bestimme den Maximum-Likelihood-Schätzer, der anhand einer Stichprobe $x_1, ...,x_n$ den Parameter $\lambda$ schätzt. (SoSe16).



\Aufgabe{4}
Für eine Zufallsvariable X mit Dichtefunktion
\begin{equation}
  f(x) = \begin{cases}
    \alpha \cdot x^{\alpha - 1} & \text{f"ur }0 < x \leq 1 \\
    0 & \text{sonst }  
   \end{cases}
\end{equation}
mit $\alpha > 0$ ergab eine Stichprobe die Werte $x_1, ...., x_n$. Bestimmen Sie einen Maximum-Likelihood Schätzer für $\alpha$. (WiSe14/15)

\Aufgabe{5}
Eine Zufallsvariable X sei stetig verteilt mit der von einem Parameter $\alpha > 0$ abhängigen Dichtefunktion
\begin{equation}
  f(x) = \begin{cases}
    \frac{2x}{\alpha}e^{-\frac{x^2}{\alpha} }& \text{f"ur } x > 0 \\
    0 & \text{sonst }  
   \end{cases}
\end{equation}

Bestimmen Sie den Maximum-Likelihood-Schätzer, der aus einer Zufallsstichprobe $x_1, ...,x_n$ den Parameter $\alpha$ schätzt. (SoSe18)

%%%%%%%%%%%%%%%%%%%%%%%%%%%%%%%%%%%%%%%%%%%%%%%%%%%%%%
%%%%%%%%%%%%%%%%%%%%%%%%%%%%%%%%%%%%%%%%%%%%%%%%%%%%%%
\end{document}

