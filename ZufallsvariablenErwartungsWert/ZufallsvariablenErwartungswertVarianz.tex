\documentclass[a4paper,12pt]{article}
\usepackage{fancyhdr}
\usepackage{fancyheadings}
\usepackage[ngerman,german]{babel}
\usepackage{german}
\usepackage[utf8]{inputenc}
%\usepackage[latin1]{inputenc}
\usepackage[active]{srcltx}
\usepackage{algorithm}
\usepackage[noend]{algorithmic}
\usepackage{amsmath}
\usepackage{amssymb}
\usepackage{amsthm}
\usepackage{bbm}
\usepackage{enumerate}
\usepackage{graphicx}
\usepackage{ifthen}
\usepackage{listings}
\usepackage{struktex}
\usepackage{hyperref}
\usepackage{tikz}
\usetikzlibrary{positioning,automata}

%%%%%%%%%%%%%%%%%%%%%%%%%%%%%%%%%%%%%%%%%%%%%%%%%%%%%%
%%%%%%%%%%%%%% EDIT THIS PART %%%%%%%%%%%%%%%%%%%%%%%%
%%%%%%%%%%%%%%%%%%%%%%%%%%%%%%%%%%%%%%%%%%%%%%%%%%%%%%
\newcommand{\Fach}{Statistik}
\newcommand{\Name}{Robin Feldmann}
\newcommand{\Tutorium}{}
\newcommand{\Semester}{OHM}
\newcommand{\KlausurLoesung }{SoSe2020} %  <-- UPDATE ME
%%%%%%%%%%%%%%%%%%%%%%%%%%%%%%%%%%%%%%%%%%%%%%%%%%%%%%
%%%%%%%%%%%%%%%%%%%%%%%%%%%%%%%%%%%%%%%%%%%%%%%%%%%%%%

\setlength{\parindent}{0em}
\topmargin -1.0cm
\oddsidemargin 0cm
\evensidemargin 0cm
\setlength{\textheight}{9.2in}
\setlength{\textwidth}{6.0in}

%%%%%%%%%%%%%%%
%% Aufgaben-COMMAND
\newcommand{\Aufgabe}[1]{
  {
  \vspace*{0.5cm}
  \textsf{\textbf{Aufgabe #1}}
  \vspace*{0.2cm}
  
  }
}

\newcommand{\Definition}[1]{
  {
  \vspace*{0.5cm}
  \textsf{\textbf{#1}}
  \vspace*{0.2cm}
  
  }
}

%%%%%%%%%%%%%%
\hypersetup{
    pdftitle={\Fach{}: Übungsblatt \KlausurLoesung{}},
    pdfauthor={\Name},
    pdfborder={0 0 0}
}

\lstset{ %
language=java,
basicstyle=\footnotesize\tt,
showtabs=false,
tabsize=2,
captionpos=b,
breaklines=true,
extendedchars=true,
showstringspaces=false,
flexiblecolumns=true,
}

\title{Übungsblatt Deterministische Endliche Automaten}
\author{\Name{}}

\begin{document}
\thispagestyle{fancy}
\lhead{\sf \large \Fach{} \\ \small \Name{} }
\rhead{\sf \Semester{} \\  Tutorium\Tutorium{}}
\vspace*{0.2cm}
\begin{center}
\LARGE \sf \textbf{Erwartungswert und Varianz von Zufallsvariablen}
\end{center}
\vspace*{0.2cm}

%%%%%%%%%%%%%%%%%%%%%%%%%%%%%%%%%%%%%%%%%%%%%%%%%%%%%%
%% Insert your solutions here %%%%%%%%%%%%%%%%%%%%%%%%
%%%%%%%%%%%%%%%%%%%%%%%%%%%%%%%%%%%%%%%%%%%%%%%%%%%%%%
\Definition{Erwartungswert - Diskret}

$$ \mathbb{E}(x) = x_1 \cdot \mathbb{P}(x=x_1) + x_2 \cdot \mathbb{P}(x=x_2) \cdot ... = \sum_{i=1}^n x_i p_i $$

\Definition{Erwartungswert - Stetig}

$$ \mathbb{E}(x) =  \int_{-\infty}^{\infty} x\cdot f(x)dx$$

\Definition{Eigenschaften des Erwartungswerts}
Sei $h(x): \mathbb{R} \rightarrow \mathbb{R} $ eine Funktion, dann gilt: \\

Diskret: $ \mathbb{E}(h(x)) =   \sum_{i=1}^n h(x_i) p_i $  \\

Stetig: $ \mathbb{E}(h(x)) =  \int_{-\infty}^{\infty} h(x)\cdot f(x)dx $ \\

Außerdem gilt: \\

$ \mathbb{E}(h(x) + g(x)) = \mathbb{E}(h(x)) + \mathbb{E}(g(x)) $ \\


$ \mathbb{E}(aX+c) = a\mathbb{E}(X) + c $


\Definition{Varianz}
$$  \mathbb{V}(x)= \mathbb{E}((x-\mathbb{E}(x))^2) $$

Oder alternativ:

$$  \mathbb{V}(x)= \mathbb{E}(x^2) - \mathbb{E}(x)^2 $$

\Definition{Varianz - Diskret}

$$   \mathbb{V}(x)= \sum_{i=1}^n (x_i -  \mathbb{E}(x))^2  p_i $$ 

\Definition{Varianz - Stetig}

$$   \mathbb{V}(x)= \int_{-\infty}^{\infty} (x -  \mathbb{E}(x))^2 \cdot f(x)dx$$ 

\Definition{Rechenregeln}
Für X,Y unabhängige Zufallsvariablen:
$$ \mathbb{V}(aX + bY) = a^2 \cdot  \mathbb{V}(X) + b^2 \cdot  \mathbb{V}(Y) $$
$$ \mathbb{V}(X + a) = \mathbb{V}(X) $$ 

\Aufgabe{2}
Seien X und Y unabhängige Zufallsgrößen mit den Erwartungswerten $\mathbb{E}(X) = 5$, $\mathbb{E}(Y) = 3$ und den Varianzen $\mathbb{V}(X) =1$, $\mathbb{V}(Y) =2$.
Bestimmen Sie den Erwartungswert und die Varianz folgender Zufallsgrößen:
\begin{enumerate}[a)]
\item
$Z_1 =  2\cdot X $
\item
$Z_1 =  X + Y$
\item
$Z_1 =  X -  Y$
\end{enumerate} 

\Aufgabe{2}
Seien X und Y unabhängige Zufallsgrößen mit den Erwartungswerten $\mathbb{E}(X) = 10$, $\mathbb{E}(Y) = 12$ und den Varianzen $\mathbb{V}(X) =4$, $\mathbb{V}(Y) =18$.
Bestimmen Sie den Erwartungswert und die Varianz folgender Zufallsgrößen:
\begin{enumerate}[a)]
\item
$Z_1 = \frac{1}{2} X + \frac{1}{3} Y$
\item
$Z_1 = \frac{1}{2} X - \frac{1}{3} Y$
\item
$Z_1 = \frac{1}{2} X - \frac{1}{3} X$
\end{enumerate} 
(WiSe15/16)

\Aufgabe{3}
Die Zufallsvariablen $X, Y, Z, X_1, ..... X_{100}$ seien unabhängig mit dem Erwartungswert 5 und der Varianz 20.
\begin{enumerate}[a)]
\item
Berechnen Sie den Erwartungswert und die Varianz von $2\cdot X + 3 \cdot Y - Z$
\item 
Berechnen Sie den Erwartungswert und die Varianz von $S:=X_1 + .... + X_{100}$.
\end{enumerate} 
(WiSe14/15)

\Aufgabe{5}
Von einer reellen Zufallsvariable X sei bekannt, dass sie nur die Werte -1,1 und 4 annimmt, mit den Wahrscheinlichkeiten
$$ P(X=-2) = \frac{1}{4}, \quad P(X = -1) = \frac{1}{8}  \quad P(X = 0) = \frac{1}{8} \quad und \quad P(X=1)=\frac{1}{2} $$
\begin{enumerate}[a)]
\item Berechnen Sie die $\mathbb{V}(X)$. Bestimmen Sie dafür $\mathbb{E}(X)$ und $\mathbb{E}(X^2)$.
\item Skizzieren Sie die Verteilungsfunktion der Zufallsvariablen X und $ Y = -X + 1 $. Berechnen Sie $\mathbb{V}(Y)$ und $\mathbb{E}(Y)$.
\end{enumerate} 



\Aufgabe{6}
Von einer reellen Zufallsvariable X sei bekannt, dass sie nur die Werte -1,1 und 4 annimmt, mit den Wahrscheinlichkeiten
$$ P(X=-1) = \frac{5}{8}, \quad P(X = 1) = \frac{1}{8}  \quad und  \quad P(X = 4) = \frac{1}{4} $$
\begin{enumerate}[a)]
\item Berechnen Sie die $\mathbb{V}(X)$. Bestimmen Sie dafür $\mathbb{E}(X)$ und $\mathbb{E}(X^2)$.
\item Skizzieren Sie die Verteilungsfunktion der Zufallsvariablen $ Y = -2X + 3 $. Berechnen Sie $\mathbb{V}(Y)$.
\item Die Zufallsvariable Z sei unabhänging von X und habe die gleiche Verteilung. Bestimmen Sie die Wahrscheinlichkeit $P(X + Z <  0)$ sowie $\mathbb{V}(X + Z  ) $.
\end{enumerate} 
(SoSe18/19)

\Aufgabe{7}
Es sei X eine stetige Zufallsvariable, mit Dichtefunktion:
\begin{equation}
   f(x) = \begin{cases}
     3\cdot x^2, & \text{f"ur }  0 < x < 1\\
    0 & \text{sonst }  
   \end{cases}
\end{equation}
Bestimmen sie $\mathbb{E}(X)$ und $\mathbb{V}(X)$.

\Aufgabe{8}
Es sei X eine stetige Zufallsvariable, mit Dichtefunktion:
\begin{equation}
   f(x) = \begin{cases}
     x , & \text{f"ur }  0 < x < 1\\
     1 , & \text{f"ur }  1 < x < 1.5\\
    0 & \text{sonst }  
   \end{cases}
\end{equation}
Bestimmen sie $\mathbb{E}(X)$ und $\mathbb{V}(X)$.



\Aufgabe{9}
Es sei X eine stetige Zufallsvariable, mit Dichtefunktion:
\begin{equation}
   f(x) = \begin{cases}
     3\cdot x^{-4}, & \text{f"ur }  x > 1\\
    0 & \text{sonst }  
   \end{cases}
\end{equation}
Bestimmen sie $\mathbb{E}(X)$ und $\mathbb{V}(X)$.
(SoSe20)
%%%%%%%%%%%%%%%%%%%%%%%%%%%%%%%%%%%%%%%%%%%%%%%%%%%%%%
%%%%%%%%%%%%%%%%%%%%%%%%%%%%%%%%%%%%%%%%%%%%%%%%%%%%%%
\end{document}

